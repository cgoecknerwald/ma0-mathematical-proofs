\documentclass[12pt]{exam}
\usepackage{amsmath}
\usepackage{amssymb}
%Preamble here
\pagestyle{headandfoot}
\firstpageheader{Claire Goeckner-Wald}{}{24 Aug 2015}
\runningheader{Claire Goeckner-Wald}{}{}
\firstpagefooter{}{}{Page \thepage\ of \numpages}
\runningfooter{}{}{Page \thepage\ of \numpages}

%Beginning the document & entering the questions layer
\begin{document}
\begin{questions}

%templates
%\begin{parts}
%\part
%\begin{align*}
%\begin{equation*}

% QUESTION 1 ---------------------------------------------------
\question
\textbf{Proposition.} Let $m \neq 0$ and $b$ be real numbers. There exists a unique $x$ such that $mx+b=0$.
\\
\\\textbf{Discussion.} First, we will show that there exists an $x$ such that $mx+b=0$ is true. Next, we will show that this value of $x$ is unique for $mx+b=0$. To do this second step, we will consider a value $y$ that also satisfies $mx+b=0$ (replacing variable $x$ for $y$). Then, we will show that $x=y$. By showing that $x=y$, we discover that the initial value found for $x$, such that $mx+b=0$ is true, is unique.
\\
\\\textbf{Proof.} Consider the number $ x = \frac{-b}{m}$. Notice that $m(\frac{-b}{m}) +b = 0$. Thus, there exists at least one real number $x$ such that $mx+b=0$. To show that there exists a unique such $x$, we will assume that $x$ and $y$ both satisfy $mx+b=0$ (replacing variable $x$ for $y$) and show that $x=y$. So, if $mx+b=0$ and $my+b=0$, then $mx+b=0=my+b$. So, $mx+b=my+b$ and $mx=my$. Because $m \neq 0$, we can divide by $m$, to obtain that $x=y$. Thus, there exists a unique $x$ such that $mx+b=0$.
\begin{flushright}
$\square$
\end{flushright}

% QUESTION 2 ---------------------------------------------------
\question
\textbf{Proposition.} Let $x$ be a real number. $-1\leq x\leq1$ if and only if $x^2\leq1$.
\\
\\\textbf{Discussion.} The biconditional statement $-1\leq x\leq1 \Leftrightarrow x^2\leq1$ can be represented as $p\Leftrightarrow q$. This statement can be transformed into two statements, $p\Rightarrow q$ and $q\Rightarrow p$, where $p$ is that $-1\leq x\leq1$  and $q$ is that $x^2\leq1$. Notice that the second statement, $q$, has the hypothesis $x^2\leq1$, which relies upon information regarding $x^2$. Instead, to prove that $q\Rightarrow p$, we will consider its contrapositive, which is $\neg p \Rightarrow \neg q$, given by the statement ``if  $x>1$ or $x<-1 $ then $ x^2>1$". Employing the contrapositive, it is possible to prove that  $q\Rightarrow p$. By proving that $p\Rightarrow q$ and $q\Rightarrow p$ are both true statements, $p\Leftrightarrow q$ can be declared true, as desired.
\\
\\\textbf{Proof.} This biconditional statement $-1\leq x\leq1 \Leftrightarrow x^2\leq1$ will be considered as the distinct statements ``if $-1\leq x\leq1 $ then $ x^2\leq1$" and ``if $x^2\leq1$ then $ -1\leq x\leq1$". 
\par
First, we will prove that $-1\leq x\leq1\Rightarrow x^2\leq1$. Thus, we will assume that $x \geq -1$ and $x \leq1$. For ease of proof, we will separate this scenario into two cases (where $x$ still fits the initial assumption): when $x$ is negative, and when $x$ is non-negative. Therefore, we will consider two cases: when $x \geq -1$ but $x<0$, and when $x \leq1$ but $x\geq0$. In other words, we have $-1 \leq x < 0$ or $0\leq x \leq 1$.
\par
First, consider the case that $x \geq -1$ but $x<0$. We will attempt to show that $x^2\leq1$. If we multiply  $x \geq -1$ by itself, ``flipping" the inequality in the process because $x$ is some negative real number (remember,  $x<0$), we arrive at the statement that $x \cdot x \leq -1 \cdot -1$, or $x^2 \leq 1$. Thus, we have shown the first case, when  $-1 \leq x < 0$.
\par
Now, consider the case when $x \leq1$ but $x\geq0$. Again, we will multiply $x \leq1$ by itself (not flipping the inequality, because $x$ is assumed a non-negative real number, being greater or equal to $0$), to the result that $x \cdot x \leq 1 \cdot 1$, or $x^2 \leq 1$. We have now shown the second case, when $0\leq x \leq 1$. Having already shown the first case, when $-1 \leq x < 0$, we can conclude that $x^2 \leq 1$ is true for the entire range $-1\leq x\leq1$. Therefore, we can conclude that $p \Rightarrow q$ is true: if $-1\leq x\leq1$ then $x^2\leq1$.
\par
Now, we must prove $q \Rightarrow p$, that $x^2\leq1\Rightarrow -1\leq x\leq1$. Since the hypothesis of this statement provides information about $x^2$, we will use proof by contrapositive, or $\neg p \Rightarrow \neg q$. The negation of $x^2\leq1$ is $x^2>1$ and the negation of $-1\leq x\leq1$ is that $x>1$ or $ x<-1$. Therefore, the contrapositive of the initial statement is $x>1 \vee x<-1 \Rightarrow x^2>1$. This can be further segmented into $x>1\Rightarrow x^2>1$ and $x<-1 \Rightarrow x^2>1$.
\par
Let us first prove that $x>1\Rightarrow x^2>1$. Since $x>1>0$, then both $x$ and 1 are positive. Thus, we can multiply the inequality $x>1$ with itself, yielding $x \cdot x >1\cdot 1$, or $x^2>1$. Thus, if $x>1$ then $x^2>1$. 
\par
Finally, we will prove that $x<-1 \Rightarrow x^2>1$. Notice that if $x<-1$, then $x$ and $-1$ are also less than $0$ because $x<-1<0$. Thus, $x$ is negative. We can multiply $x<-1$ by itself, flipping the inequality in the process because $x$ is negative, to obtain that $x \cdot x > -1 \cdot -1$, or $x^2>1$. Therefore, if $x<-1$, then  $x^2>1$. Thus, the initial statement that $x<-1 \Rightarrow x^2>1$ is true. Together, we have shown that  $x>1 \vee x<-1 \Rightarrow x^2>1$, which is the contrapositive of the initial statement $x^2\leq1\Rightarrow -1\leq x\leq1$. Therefore, the initial statement $q \Rightarrow p$ is true as well.
\par
In summary, we have proved that the statement  $-1\leq x\leq1\Rightarrow x^2\leq1$ is true, and we have proved that the statement $x^2\leq1\Rightarrow -1\leq x\leq1$ is true via contrapositive. Therefore, we have proved that the biconditional statement $-1\leq x\leq1 \Leftrightarrow x^2\leq1$ is true, as desired.
\begin{flushright}
$\square$
\end{flushright}

% QUESTION 3 ---------------------------------------------------
\question
\textbf{Proposition.} Let $m$ and $n$ be whole numbers. The values $m$ and $n$ have the same parity if and only if $m+n$ is even.
\\
\\\textbf{Discussion.} The biconditional statement ``$m$ and $n$ have the same parity $\Leftrightarrow m+n$ is even" can be represented as $p\Leftrightarrow q$. This statement can be transformed into two statements: the first being that $p\Rightarrow q$ and the second being that $q\Rightarrow p$. Notice that the second statement, $q$, has the hypothesis ``$m+n$ is even", which relies upon information regarding the parity of $m+n$. Instead, to prove that $q\Rightarrow p$, we will consider its contrapositive, which is $\neg p \Rightarrow \neg q$. Employing the contrapositive, it is possible to prove that  $q\Rightarrow p$. By proving that $p\Rightarrow q$ and $q\Rightarrow p$ are both true statements, $p\Leftrightarrow q$ can be declared true, as desired. 
\par
Remember that a number $x$ is understood to be even if it can be represented as $x=2k$ for some $k \in \mathbb Z$; similarly, the number is odd if it can be written as $x=2k+1$ for some  $k \in \mathbb Z$.\\
\\\textbf{Proof.} Assume that $m$ and $n$ are both whole numbers. We will break the biconditional statement ``$m$ and $n$ have the same parity $\Leftrightarrow$  $m+n$ is even" into two statements. The first is that  ``$m$ and $n$ have the same parity $\Rightarrow$ $m+n$ is even" and the second is that ``$m+n$ is even $\Rightarrow$ $m$ and $n$ have the same parity".
\par 
Let us first prove the statement ``$m$ and $n$ have the same parity $\Rightarrow$  $m+n$ is even". Two numbers are said to have the same parity if they are both even or both odd. Thus, the statement can be rewritten as ``$m$ and $n$ are both even $\Rightarrow$  $m+n$ is even" and ``$m$ and $n$ are both odd $\Rightarrow$ $m+n$ is even".  Thus, assuming that $m$ and $n$ are both even, $m+n$ can be written as $2k+2j$ for some $k,j \in \mathbb Z$. Isolating the $2$, we have that if $m$ and $n$ are both even, then they can be understood as $m+n = 2(k+j)$. Because $k$ and $j$ are contained in the set of integers $\mathbb Z$, and $k+j$ is multiplied by $2$, by the definition of even parity, it follows that $m+n$ is even. Thus, we have ``if $m$ and $n$ are both even then  $m+n$ is even". 
\par
Similarly, assuming that $m$ and $n$ are both odd, then $m+n$ can be written as $(2k+1)+(2j+1)$ for $k,j \in \mathbb Z$, by the definition of odd parity. We can distribute to write that
\begin{align*}
m+n &= 2k+1+2j+1; \\
m+n &= 2k + 2j+2; \\
m+n &= 2(k+j+1). \\
\end{align*}
Since $k$, $j$, and $1$ are all contained in the set of integers $\mathbb Z$, which is closed under addition, then their sum is also contained in  $\mathbb Z$. Since this sum $k+j+1$ is multiplied by $2$ to get the sum of the odd numbers $m$ and $n$, by the definition of even parity, this sum $m+n$ is even. Therefore, it is true that ``$m$ and $n$ are both odd $\Rightarrow$ $m+n$ is even". Thus, the entire statement ``$m$ and $n$ have the same parity $\Rightarrow$  $m+n$ is even" is true.
\par
Now we will prove the second statement ``$m+n$ is even $\Rightarrow$ $m$ and $n$ have the same parity". Because this statement has the hypothesis ``$m+n$ is even", which relies upon information regarding the parity of $m+n$, to prove the statement, we will use its contrapositive. Remember that a statement $q\Rightarrow p$ has the contrapositive of $\neg p \Rightarrow \neg q$. Thus, the contrapositive of the statement ``$m+n$ is even $\Rightarrow$ $m$ and $n$ have the same parity" is ``$m$ and $n$ have different parities $\Rightarrow$ $m+n$ is odd". Assuming that $m$ and $n$ have different parities, $m+n$ can be written as $(2k+1)+(2j)$ for $k,j \in \mathbb Z$, irrespective of which number is of which parity without loss by generalization. We know that $m+n=(2k+1)+(2j)=2(k+j) + 1$, where $k$, $j$, and thus their sum, are contained in $\mathbb Z$. By the definition of odd parity, it becomes apparent that $m+n = 2(k+j) + 1$ is some odd number where $k+j \in \mathbb Z$. Thus, it is true that ``if $m$ and $n$ have different parities then $m+n$ is odd". We have proved the contrapositive of the initial statement. Therefore, the original statement ``$m+n$ is even $\Rightarrow$ $m$ and $n$ have the same parity" is true.
\par
In summary, we have proved that ``$m$ and $n$ have the same parity $\Rightarrow$  $m+n$ is even" is true, and we have proved that ``$m+n$ is even $\Rightarrow$ $m$ and $n$ have the same parity" is true. Thus, we have that ``$m$ and $n$ have the same parity $\Leftrightarrow$  $m+n$ is even", as desired.
\begin{flushright}
$\square$
\end{flushright}

% QUESTION 4 ---------------------------------------------------
\question
\textbf{Proposition.} Let $m$ and $n$ be whole numbers. If $m\cdot n$ is odd, then $m$ and $n$ are both odd.
\\
\\\textbf{Discussion.} The statement ``$m\cdot n$ is odd $\Rightarrow$ $m$ and $n$ are both odd" can be represented by $p\Rightarrow q$. We will use the technique of proof by contrapositive to prove this statement true. The contrapositive of $p\Rightarrow q$ is $\neg q \Rightarrow \neg p$. Therefore, the contrapositive of the initial statement is ``$m$ is even $\vee$ $n$ is even $\Rightarrow$ $m\cdot n$ is even". For the statement ``$m$ is even $\vee$ $n$ is even" to be true, there are two possible states for $m$ and $n$. The first is that one of the numbers is even; the second is that both numbers are even. We will prove the statement true for both states. Thus, by finding the contrapositive to be true, we will have found that the initial statement is true, as desired.
\par
Remember that a number $x$ is understood to be even if it can be represented as $x=2k$ for some  $k \in \mathbb Z$; similarly, the number is odd if it can be written as $x=2k+1$ for some $k \in \mathbb Z$. 
\\
\\\textbf{Proof.} We will prove the statement ``$m\cdot n$ is odd $\Rightarrow$ $m$ and $n$ are both odd" true by proving its contrapositive true. As noted above, the contrapositive of the original statement is ``if $m$ is even or $n$ is even then $m\cdot n$ is even". As discussed, for the statement ``$m$ is even $\vee$ $n$ is even" to be true, either one of the numbers is even, or both are even.
\par
We will start with the assumption that only one number of the two is even. In this case, by representing the even number as $2k$ and by representing the odd number as $2j+1$, for $k,j \in \mathbb Z$. We can see that the value of $m \cdot n$ can be represented by $2k(2j+1)$ Thus, $m \cdot n$ is equivalent to $4kj + 2k$, or $2(2kj+k)$. Because $k$, $j$, and $2$ are contained in the set of integers $\mathbb Z$, which is closed under addition and multiplication, then we know that $2kj+k$ is also in $\mathbb Z$. Since we have that $m \cdot n = 2(2kj+k)$, by the definition of even parity, we can conclude that  $m \cdot n$ is an even number. We have proved that if either $m$ or $n$ is even (but not yet both), then $m\cdot n$ is even.
\par 
Now we will assume that both $m$ and $n$ are even. By substituting $m$ and $n$, respectively, for $2k$ and $2j$, where $k$ and $j$ are still some integers contained in $\mathbb Z$, we can see that $m\cdot n$ can be represented as $2k \cdot 2j$, or $4kj$. This can be understood as $m \cdot n = 2(2kj)$ where $m$ and $n$ are both even. Once again,   because $k$, $j$, and $2$ are contained in the set of integers $\mathbb Z$, which is closed under multiplication, we know that $2kj \in \mathbb Z$. By the definition of even parity, since $m \cdot n = 2(2kj)$, we can conclude that the product $m \cdot n$ is even. Thus, $m \cdot n$ is an even number when both $m$ and $n$ are even. 
\par
In summary, we have proved the contrapositive statement that ``$m$ is even $\vee$ $n$ is even $\Rightarrow$ $m\cdot n$ is even" is true, for the cases when both $m$ and $n$ are even, and when only one of the two is even. By proving this statement true, we have also proved the initial statement ``$m\cdot n$ is odd $\Rightarrow$ $m$ and $n$ are both odd" true, as desired.
\begin{flushright}
$\square$
\end{flushright}

% QUESTION 5 ---------------------------------------------------
\question
\begin{parts}
\part First, the representation of $n = -3, -1, 1, 3, 5, 7, 9$ as the difference of two perfect squares, as requested.
\begin{align*}
-3&=1-4=1^2-2^2\\
-1&=0-1=0^2-1^2\\
1&=1-0=1^2-0^2\\
3&=4-1=2^2-1^2\\
5&=9-4=3^2-2^2\\
7&=16-9=4^2-3^2\\
9&=25-16=5^2-4^2
\end{align*}
\part
\textbf{Proposition.} Every odd whole number can be written as the difference of two perfect squares.
\\
\\\textbf{Discussion.} A perfect square, sometimes called a square number, is an integer that is the square of an integer. All square numbers are non-negative due to the effect that occurs when an integer is squared, or multiplied by itself. Moreover, an odd whole number $n$ is one that can be represented as $n=2k+1$, where $k$ is some integer contained in $\mathbb Z$. For each odd whole number $n$, we will represent $n$ as $(k+1)^2-k^2$, or the difference of two perfect squares.
\\
\\\textbf{Proof.} First, we will represent an odd whole number $n$ as $2k+1$, where $k$ is some integer contained in the set $\mathbb Z$. Because $k^2-k^2$ is equivalent to $0$, we can add $k^2-k^2$ to the right-hand side of the equation, writing that:
\begin{align*}
n &=2k+1+(k^2-k^2) ;\\
n &=2k+1+k^2-k^2 ;\\
n &=k^2+ 2k+1-k^2 ;\\
n &=(k^2+ 2k+1)-k^2 ;\\
n &=(k+1)^2-k^2. 
\end{align*}
We have reached $n=(k+1)^2-k^2$. Note that because $k$ and $1$ are contained in $\mathbb Z$, which is closed under addition, we have that $k+1$ is also contained in $\mathbb Z$. Thus, since $n=2k+1$ by the definition of odd parity, then we have $n = (k+1)^2-k^2$. Recall that by the definition of a perfect square $(k+1)^2$ and $k^2$ are both perfect squares. In words, any odd whole number $n$ such that $n=2k+1$, where $k$ is some integer, can also be understood as $n=(k+1)^2-k^2$, which is the difference of two perfect squares, as desired.

\end{parts}
\begin{flushright}
$\square$
\end{flushright}


%Conclusion
\end{questions}
\end{document}
