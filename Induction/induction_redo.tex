\documentclass[12pt]{exam}
\usepackage{amsmath}
\usepackage{amssymb}
%Preamble here
\pagestyle{headandfoot}
\firstpageheader{Claire Goeckner-Wald}{}{27 Aug 2015}
\runningheader{Claire Goeckner-Wald}{}{}
\firstpagefooter{}{}{Page \thepage\ of \numpages}
\runningfooter{}{}{Page \thepage\ of \numpages}

%Beginning the document & entering the questions layer
\begin{document}
\begin{questions}

%templates
%\begin{parts}
%\part
%\begin{align*}
%\begin{equation*}

% QUESTION 1 ---------------------------------------------------
\question
\textbf{Proposition.} Let $r \neq 1$ be a real number. Using mathematical induction, $$\sum_{j=0}^n r^j = \frac{1-r^{n+1}}{1-r}.$$
\textbf{Proof.} Assume $r \neq 1$ is contained in the set of real numbers $\mathbb R$. Let the statement $A(n)$ be given by $$\sum_{j=0}^n r^j = \frac{1-r^{n+1}}{1-r}.$$ We will use mathematical induction to show that $A(n)$ is true for all $n\geq 0$. First we confirm that the base case A(0) is true. We have that $$\sum_{j=0}^0 r^j = r^0 = 1.$$ Moreover, since $r \neq 1$, we can say that $$\frac{1-r^{0+1}}{1-r} =\frac{1-r}{1-r} = 1.$$ Thus A(0) is true because both sides of the equation are equal to one; thus, they are equivalent. Next, we perform the inductive step. We assume that $A(n)$ is true. In other words, we assume that $$\sum_{j=0}^n r^j = \frac{1-r^{n+1}}{1-r} $$ for some $n\geq 0$. We will use this assumption to prove that $A(n+1)$ is true by showing that $$\sum_{j=0}^{n+1} r^j = \frac{1-r^{(n+1)+1}}{1-r}. $$
Beginning with the left-hand side of the $A(n+1)$ statement and using our inductive assumption, we have that $$\sum_{j=0}^{n+1} r^j = \Bigg[ \sum_{j=0}^{n} r^j \Bigg] + r^{n+1}.$$ Substituting $\sum_{j=0}^n r^j$ for $ \frac{1-r^{n+1}}{1-r} $ (as they are equivalent via the inductive assumption), we then write that $$\sum_{j=0}^{n+1} r^j =  \frac{1-r^{n+1}}{1-r} + r^{n+1}.$$ Using basic algebra, we can manipulate the right-hand side of the equation to write that $$\sum_{j=0}^{n+1} r^j =  \frac{1-r^{n+1} + (1-r)r^{n+1}}{1-r}.$$ Then, foiling and simplifying, we discover that $$\sum_{j=0}^{n+1} r^j =  \frac{1-r^{n+1} + r^{n+1} -r\cdot r^{n+1}}{1-r}.$$ Canceling the addition and subtraction of the value $r^{n+1}$, we then write that:
\begin{align*}
\sum_{j=0}^{n+1} r^j &=  \frac{1 -r\cdot r^{n+1}}{1-r}; \\
\sum_{j=0}^{n+1} r^j &=  \frac{1 -r^{(n+1)+1}}{1-r}.
\end{align*}
Thus we have shown that $A(n+1)$ is true. By induction, we can now conclude that $A(n)$ given by $$\sum_{j=0}^n r^j = \frac{1-r^{n+1}}{1-r} $$ where $r \neq 1$ is a real number, is true for all $n$ greater than the base case $n=0$, as desired.
\begin{flushright}
$\square$
\end{flushright}

% QUESTION 2 ---------------------------------------------------
\question 
Consider the function $f(x) = \dfrac{1}{1-x}.$ 
\begin{parts}
% QUESTION 2 PART A
\part
Compute the first several derivatives of $f$ and conjecture a pattern for $f^{(n)}(x)$.
\bigskip
\par
First, to ease integration we will rewrite $f(x)$ as $$f(x) = \dfrac{1}{1-x} = \dfrac{-1}{-(1-x)} = \dfrac{-1}{x-1} = (-1)(x-1)^{-1}.$$ The first several derivatives of the function $f$ follow:
\begin{align*}
f(x) &= (-1)(x-1)^{-1} \\
f'(x) &=  (1)(x-1)^{-2} \\
f''(x) &=  (-2)(x-1)^{-3} \\
f^{(3)}(x) &= (6)(x-1)^{-4} \\
f^{(4)}(x) &=  (-24)(x-1)^{-5} .\\
\end{align*}
The conjectured pattern is $$f^{(n)}(x) = (-1)^{n+1}(n!)(x-1)^{-(n+1)}.$$ 
% QUESTION 2 PART B
\part
\textbf{Proposition.} The conjectured pattern is true for $f^{(n)}(x)$. 
\\
\\\textbf{Proof.}  Let $A(n)$ be the statement $$f^{(n)}(x) = (-1)^{n+1}(n!)(x-1)^{-(n+1)}.$$ We wish to show that $A(n)$ is true for all integers $n \geq 0$. We must verify that the base case $A(0)$ is true, which says that $$f^{(0)}(x) = (-1)^{0+1}(0!)(x-1)^{-(0+1)}.$$ Since $f^{(0)}$ is simply the original function, and $$f^{(0)}(x) = (-1)^{1}(1)(x-1)^{-1} = (-1)(x-1)^{-1} = \dfrac{-1}{x-1} =  \dfrac{1}{1-x},$$ which is also the original function, then the base case $A(0)$ is true. We will assume for some $n \geq 0$ that $$f^{(n)}(x) = (-1)^{n+1}(n!)(x-1)^{-(n+1)}$$ is true. This is the inductive assumption. We will then show $A(n+1)$ is also true by showing $$f^{(n+1)}(x) = (-1)^{(n+1)+1}((n+1)!)(x-1)^{-((n+1)+1)}.$$ We will do so by writing the $(n+1)$st derivative as the derivative of the $n$-th derivative of function $f$. We will use basic techniques of derivation. To start, 
\begin{align*}
f^{(n+1)}(x) &= \dfrac{d}{dx} f^{(n)}(x) \\
&= \dfrac{d}{dx} \Big[ (-1)^{n+1}(n!)(x-1)^{-(n+1)} \Big] \\
&= (-1)^{n+1}(n!) \cdot \dfrac{d}{dx} \Big[ (x-1)^{-(n+1)} \Big] .\\
\end{align*}
Then, we take the derivative of $(x-1)^{-(n+1)}$ for the statement $$f^{(n+1)}(x) = (-1)^{n+1}(n!) \cdot \Big[ -(n+1)(x-1)^{-(n+1) -1} \Big].$$ Simplifying using basic algebra, we write 
\begin{align*}
f^{(n+1)}(x) &= (-1)^{n+1}(n!) \cdot \Big[ -(n+1)(x-1)^{-(n+1) -1} \Big] .\\
&= (-1)^{n+1}(n!) (-1)(n+1)(x-1)^{-((n+1)+1)} .\\
\end{align*}
Because $(-1)^{n+1}$ multiplied by $ -1 $ is equal to $(-1)^{n+2}$ we write that $$f^{(n+1)}(x) = (-1)^{n+2}(n!) (n+1)(x-1)^{-((n+1)+1)}.$$ Then, because $n!$ multiplied by $n+1$ can be written as $(n+1)!$ we write that $$f^{(n+1)}(x) = (-1)^{n+2}(n+1)!(x-1)^{-((n+1)+1)}.$$ We have therefore shown our statement $A(n+1)$ to be true, and our inductive step is complete. By induction, we know that the statement $A(n)$ given by $$f^{(n)}(x) = (-1)^{n+1}(n!)(x-1)^{-(n+1)}$$ is true for all $n\geq 0$, as desired.
\begin{flushright}
$\square$
\end{flushright}
\end{parts}

% QUESTION 3 ---------------------------------------------------
\question
\textbf{Proposition.} Let $x > -1$. The statement $$(1+x)^n \geq 1+nx$$ is true for all integers $n \geq 1.$ 
\\
\\\textbf{Proof.} Assume an $x$ greater than $-1$. We will use proof by induction to show the statement $A(n)$ given by $(1+x)^n \geq 1+nx$ is true for all integers $n \geq 1$. Thus we have our base case $n=1$. The statement $A(1)$ reads that
\begin{align*}
(1+x)^1 &\geq 1+1x;\\
1+x &\geq 1+x. 
\end{align*}
The above is a true statement because $1+x = 1+x$. Thus $A(1)$ is true. Next we perform the inductive step. Thus we assume $A(n)$ is true for some $n$ greater than the base case of $n=1$. In other words, we assume  $$(1+x)^n \geq 1+nx $$ for $n \geq 1$. We wish to show $A(n+1)$, that $$(1+x)^{n+1} \geq 1+(n+1)x.$$ We will begin on the left-hand side with $$(1+x)^{n+1} = (1+x)^n \cdot (1+x).$$ Because we have assumed that $x$ is greater than $-1$, it follow that $1+x$ is greater than $0$. Thus we can multiply both sides of the initial statement $A$ by the value $1+x$ to reveal that $$(1+x)^n \cdot (1+x) \geq (1+nx) \cdot (1+x).$$ By combining the previous two statements, we then write that $$(1+x)^{n+1} = (1+x)^n \cdot (1+x) \geq (1+nx) \cdot (1+x).$$ In foiling the right-hand side of this inequality, we have that 
\begin{align*}
(1+x)^{n+1} &\geq (1+nx) \cdot (1+x) ;\\
(1+x)^{n+1} &\geq 1 + nx + x + nx^2 ;\\
(1+x)^{n+1} &\geq 1 + (n+1)x + nx^2 .
\end{align*}
Because $n$ and $x^2$ are both greater than or equal to $0$, it follows that $nx^2 \geq 0$ as well. Thus, we can write that $$1 + (n+1)x + nx^2 \geq 1 + (n+1)x.$$ Therefore, by combination of the previous two inequalities, we can write $$(1+x)^{n+1} \geq 1 + (n+1)x + nx^2 \geq 1 + (n+1)x.$$ Elimination of the middle value reveals the original $A(n+1)$ statement:  $$(1+x)^{n+1} \geq 1 + (n+1)x.$$

Thus, we have that $A(n+1)$ is a true statement. By induction, we can now conclude that the statement $A(n)$ given by $(1+x)^n \geq 1+nx$ is true for all integers $n \geq 1$, as desired.
\begin{flushright}
$\square$
\end{flushright}

%Conclusion
\end{questions}
\end{document}
