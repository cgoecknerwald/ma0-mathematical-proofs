\documentclass[12pt]{exam}
\usepackage{amsmath}
\usepackage{amssymb}

%Preamble here
\pagestyle{headandfoot}
\firstpageheader{Claire Goeckner-Wald}{}{8 Aug 2015}
\runningheader{Claire Goeckner-Wald}{}{}
\firstpagefooter{}{}{Page \thepage\ of \numpages}
\runningfooter{}{}{Page \thepage\ of \numpages}

%Beginning the document & entering the questions layer
\begin{document}
\begin{questions}

%templates
%\begin{parts}
%\part
%\begin{align*}
%\begin{equation*}

% QUESTION 1 ---------------------------------------------------
\question
\textbf{Proposition.} The two angle-sum formulae hold:
\begin{align*}
\sin (\alpha + \beta) &= \sin\alpha\cos\beta + \sin\beta\cos\alpha ; \\
\cos(\alpha + \beta) &= \cos\alpha\cos\beta - \sin\alpha\sin\beta.
\end{align*}
\\
\\\textbf{Proof.} Euler's equation states that $e^{i\theta} = \cos{\theta} + i\sin(\theta)$. Consider the value $e^{i(\alpha + \beta)}$. From Euler's equation we know that $e^{i(\alpha + \beta)} = \cos(\alpha + \beta) + i\sin(\alpha + \beta)$. Because complex exponents follow the same rules as real exponents, we also know that $e^{i(\alpha + \beta)} = e^{i\alpha} \cdot e^{i\beta}$. Further expansion via Euler's equation reveals that $e^{i\alpha} \cdot e^{i\beta} = (\cos\alpha + i\sin\alpha)(\cos\beta + i\sin\beta)$. Expanding this result shows 
\begin{equation*}
e^{i\alpha} \cdot e^{i\beta} = \cos\alpha\cos\beta + i(\sin\alpha\cos\beta) + i(\sin\beta\cos\alpha) - \sin\alpha\sin\beta.
\end{equation*}
By rearranging this equivalence to separate the real and imaginary parts, we can see that 
\begin{equation*}
e^{i\alpha} \cdot e^{i\beta} = (\cos\alpha\cos\beta - \sin\alpha\sin\beta) + i(\sin\alpha\cos\beta + \sin\beta\cos\alpha) .
\end{equation*}
Thus, by equating the Euler expansions on the equivalent values $e^{i(\alpha + \beta)}$ and $e^{i\alpha} \cdot e^{i\beta}$, we have that 
\begin{equation*}
\cos(\alpha + \beta) + i\sin(\alpha + \beta) = (\cos\alpha\cos\beta - \sin\alpha\sin\beta) + i(\sin\alpha\cos\beta + \sin\beta\cos\alpha) .
\end{equation*}
For two complex numbers $x,y \in \mathbb{C}$ to be equivalent, their real parts and their imaginary parts must be equal. In other words, Re($x$) = Re($y$) and Im($x$) = Im($y$) must be true. Therefore, because we have equated the Euler expansions as above, we have that
 \begin{align*}
\cos(\alpha + \beta) &= \cos\alpha\cos\beta - \sin\alpha\sin\beta \\
\sin(\alpha + \beta) &= \sin\alpha\cos\beta + \sin\beta\cos\alpha.
\end{align*}
These are the double-angle formulae, as desired.
\begin{flushright}
$\square$
\end{flushright}

% QUESTION 2 ---------------------------------------------------
\question
\begin{parts}
% QUESTION 2 PART A
\part
\textbf{Proposition.}  Show $|z| = \textrm{Re}(z)$ if and only if $z$ is a non-negative real number.
\\
\\\textbf{Proof.} Here we have a biconditional statement $p \Leftrightarrow q$, where $p$ is that $|z| = \textrm{Re}(z)$ and $q$ is that $z$ is a non-negative real number. We will show $p \Rightarrow q$ as well as $q \Rightarrow p$.
\par
First we will demonstrate $p\Rightarrow q$ using proof by contrapositive $\neg q \Rightarrow \neg p$. The contrapositive is that if $z$ is negative or $z$ is not a real number, then $|z| \neq \textrm{Re}(z)$. Thus $\neg q$ has two cases: when $z$ is negative, and when $z$ is not a real number (i.e., has an imaginary component). 
\par
For the first case, let us consider a $z < 0$ contained in $\mathbb{R}$. Thus, in the Cartesian representation $z = a + bi$, we assume $a$ is some negative real number and $b=0$. Then we can write that 
\begin{equation*}
|z| = |a+bi|.
\end{equation*}
Using the definition of modulus, we have that 
\begin{equation*}
|a+bi| = \sqrt{a^2 + b^2} = \sqrt{a^2 + 0^2} = \sqrt{a^2} = |a|.
\end{equation*}
Additionally, we know that $\textrm{Re}(z) = a$, where $a$ is some negative number less than $0$. However, $|a|$ must be some nonnegative number by the properties of absolute values. Thus, $|a| \neq a$. By extension, $|z| \neq \textrm{Re}(z)$ in the case that $z$ is some negative real number.
\par
Now, consider the second case that $z$ has some imaginary part not equal to $0$. Then in Cartesian representation, $z = a + bi$, we know that $b \neq 0$. Thus, by the definition of modulus, $|z| = |a+bi| = \sqrt{a^2 + b^2}$. We know that $\textrm{Re}(z) = a$. However, in the case that $b \neq 0$, then $a \neq \sqrt{a^2 + b^2}$. Thus, when $z$ has some imaginary part, $|z| \neq \textrm{Re}(z)$. 
\par
Having shown the contrapositive $\neg q \Rightarrow \neg p$ is true for both cases, we conclude that $p \Rightarrow q$ is true as well.
\par
Now we must show $q \Rightarrow p$, which states that if $z$ is a non-negative real number, then $|z| = \textrm{Re}(z)$. Assume some real non-negative $z$. Using Cartesian form, $z = a +bi$, we know that $\textrm{Re}(z) = a$ and $a \geq 0$. Furthermore, because $z \in \mathbb{R}$, then $b = 0$. 
\par
Consider that the modulus $|z|$. We write that  
\begin{equation*}
|z| = |a+bi| = \sqrt{a^2 + b^2} = \sqrt{a^2 + 0^2} = \sqrt{a^2} = |a|.
\end{equation*} 
Because $a \geq 0$, the real part $\textrm{Re}(z) = a$ is non-negative, as well. Therefore, $|a| = a$. Thus, since $|z| = a$ and $\textrm{Re}(z) = a$, we can conclude that $|z| = \textrm{Re}(z)$ when $z$ is a non-negative real number. We have thus shown that $q \Rightarrow p$, in addition to previously showing $p \Rightarrow q$. Therefore, $p \Leftrightarrow q$ is true. We have shown that $|z| = \textrm{Re}(z)$ if and only if $z$ is a non-negative real number, as desired.
\begin{flushright}
$\square$
\end{flushright}
% QUESTION 2 PART B
\part
\textbf{Proposition.} Show that $\left(\overline z\right)^2 = z^2$ if and only if $z$ is purely real or purely imaginary (i.e., its real part is $0$). 
\\
\\\textbf{Proof.}  Here we have a biconditional statement $p \Leftrightarrow q$, where $p$ is that  $\left(\overline z\right)^2 = z^2$  and $q$ is that $z$ is purely real or purely imaginary. For $q$ to be true, either $\textrm{Re}(z) = 0$ or $\textrm{Im}(z) = 0$. We will show $p \Rightarrow q$ and $q \Rightarrow p$ separately.
\par
Let us begin with $p \Rightarrow q$ by proof of contrapositive $\neg q \Rightarrow \neg p$. Written, the contrapositive states that if $\textrm{Im}(z) \neq 0$ and $\textrm{Re}(z) \neq 0$, then $\left(\overline z\right)^2 \neq z^2$. As such, consider a $z \in \mathbb{C}$. We will represent $z$ in Cartesian form as $z = a + bi$ where $\textrm{Im}(z) \neq 0$ and $\textrm{Re}(z) \neq 0$. Thus $a\neq 0$ and $b \neq 0$. Then, we have that
\begin{equation*}
z^2 = z \cdot z = (a+bi)(a+bi).
\end{equation*}
Foiling, we find
\begin{equation*}
z^2 = a^2 + i(2ab) - b^2.
\end{equation*}
Now, consider the conjugate of $z$, given by $\overline{z} = a - bi$. Squaring, we find
\begin{equation*}
\left(\overline z\right)^2 = \overline z \cdot \overline z = (a-bi)(a-bi).
\end{equation*}
Foiling the result, we see that  
\begin{equation*}
\left(\overline z\right)^2 = a^2 - i(2ab) -b^2.
\end{equation*}
When $a \neq 0$ and $b \neq 0$, as is the case, then 
\begin{align*}
z^2 &= a^2 + i(2ab) + b^2 ;\\
\left(\overline z\right)^2 &= a^2 - i(2ab) -b^2,
\end{align*}
 are not always equivalent. Therefore, we have shown the contrapositive $\neg q \Rightarrow \neg p$: if $\textrm{Im}(z) \neq 0$ and $\textrm{Re}(z) \neq 0$, then $\left(\overline z\right)^2 \neq z^2$. Therefore, we conclude that $p \Rightarrow q$, as desired.
\par
Now, we will show $q \Rightarrow p$. Thus, if $\textrm{Im}(z) = 0$ or $\textrm{Re}(z) = 0$, then $\left(\overline z\right)^2 = z^2$. In the statement $q$, we have two cases: when $\textrm{Im}(z) = 0$, and when $\textrm{Re}(z) = 0$. 
\par
Let us consider the case when $\textrm{Im}(z) = 0$. Thus, in Cartesian representation $z = a + bi$, we assume that $b$ is zero. Therefore, $z = a + bi = a + 0i = a$. Now, consider the conjugate $\overline z = a - bi$. Again, since $\textrm{Im}(z) = 0$, we know that $b=0$ and $\overline z = a - bi = a - 0i = a$. Thus, $\overline z = a = z$ and $\overline z = z$. Squaring both sides, we see that $\left(\overline z\right)^2 = z^2$. Therefore, when $\textrm{Im}(z) = 0$, it is true that $\left(\overline z\right)^2 = z^2$.
\par 
Now, let us consider the second case when $\textrm{Re}(z) = 0$. Then, in Cartesian representation $z = a + bi$, we assume that $a$ is zero. Therefore, $z = a + bi = 0 + bi = bi$. Squaring both sides, 
\begin{equation*}
z^2 = (bi)^2 = b^2 \cdot i^2 = b^2 \cdot -1 = -b^2. 
\end{equation*}
In short, $z^2 = -b^2$. Now consider the conjugate $\overline z = a - bi$. Once again, since Re(z) = 0, we know that $a = 0$. Then, $\overline z = a - bi = 0 - bi = -bi$. Once again squaring both sides of the equation, we get that 
\begin{equation*}
\left(\overline z\right)^2 = (-bi)^2 = (-b)^2 \cdot i^2 = b^2 \cdot -1 = -b^2. 
\end{equation*}
Therefore, $\left(\overline z\right)^2$ is also equivalent to $-b^2$. We now see that in the case when $\textrm{Re}(z) = 0$, it is true that $\left(\overline z\right)^2 = z^2$, because they are both equivalent to $-b^2$.
\par
Therefore, we have shown both cases: if $\textrm{Re}(z) = 0$ or $\textrm{Im}(z) = 0$, then $\left(\overline z\right)^2 = z^2$. Thus, we have proven that  $q \Rightarrow p$. Having already shown that $p \Rightarrow q$, we can now conclude that $p \Leftrightarrow q$. In other words, $\left(\overline z\right)^2 = z^2$ if and only if $z$ is purely real or purely imaginary, as desired.
\begin{flushright}
$\square$
\end{flushright}
\end{parts}

% QUESTION 3 ---------------------------------------------------
\question
\begin{parts}
% QUESTION 3 PART A
\part
\textbf{Proposition.} If $z, w \in \mathbb C$, then $$|z \cdot w| = |z| \cdot |w|$$ using the Cartesian form $z = a+bi$ and $w = c+di$ for the complex numbers $z$ and $w$.
\\
\\\textbf{Proof.} Assume a $z,w \in \mathbb C$. Using the Cartesian representation $z = a+bi$ and $w = c+di$, we will show that $|z \cdot w| = |z| \cdot |w|$ because they are both equivalent to $\sqrt{a^2c^2 + a^2d^2 + b^2c^2 + b^2d^2}$ . 
\par
First, consider $|z \cdot w|$. Substitute the above Cartesian forms to see that $|z \cdot w|$ can be represented as $|(a + bi )(c + di)|$. Foiling, we reveal that 
\begin{equation*}
|z \cdot w| = |(a + bi )(c + di)| = |ac - bd +i(bc + ad)|
\end{equation*}
where $ac - bd$ is the real part and $bc + ad$ is the imaginary part. By the definition of modulus, we can take the square root of the sum of the squared real part and the squared imaginary part to write that
\begin{equation*}
|z \cdot w| = |ac - bd +i(bc + ad)| = \sqrt{(ac - bd)^2 + (bc + ad)^2}.
\end{equation*}
Foiling once again, we can show that
\begin{equation*}
|z \cdot w| = \sqrt{(ac - bd)^2 + (bc + ad)^2} = \sqrt{a^2c^2 + a^2d^2 + b^2c^2 + b^2d^2}.
\end{equation*}
Now, consider $|z| \cdot |w|$. In Cartesian form, we represent this value as $|a+bi|\cdot |c+di|$. Thus, employing the definition of modulus, we reveal that
\begin{equation*}
|z| \cdot |w| = |a+bi|\cdot |c+di| = \sqrt{a^2 + b^2} \cdot \sqrt{c^2 + d^2}.
\end{equation*}
Using basic algebraic laws, we can further rewrite this equation to show that
\begin{equation*}
|z| \cdot |w| = \sqrt{a^2 + b^2} \cdot \sqrt{c^2 + d^2} = \sqrt{(a^2 + b^2)(c^2 + d^2)}.
\end{equation*}
Foiling this equivalence reveals to us that 
\begin{equation*}
|z| \cdot |w| = \sqrt{a^2c^2 + a^2d^2 + b^2c^2 + b^2d^2}.
\end{equation*}
Therefore, we have shown that $|z \cdot w|$ and  $|z| \cdot |w|$ are both equivalent to the value  $\sqrt{a^2c^2 + a^2d^2 + b^2c^2 + b^2d^2}$. Thus, they must be also equivalent. In other words, $|z \cdot w| = |z| \cdot |w|$ for complex numbers $z$ and $w$, as desired.
\begin{flushright}
$\square$
\end{flushright}
% QUESTION 3 PART B
\part
\textbf{Proposition.} If $z, w \in \mathbb C$, then $$|z \cdot w| = |z| \cdot |w|$$ using the polar form $z = r_1e^{i\theta_1}$ and $w = r_2e^{i \theta_2}$ for the complex numbers $z$ and $w$.  
\\
\\\textbf{Proof.} Now we will demonstrate the same phenomena $|z \cdot w| = |z| \cdot |w|$ using complex polar  representation such that  $z = r_1e^{i\theta_1}$ and $w = r_2e^{i \theta_2}$. We will show that $|z \cdot w|$ and $ |z| \cdot |w|$ are equivalent because they are both equal to the value $r_1 r_2$.
\par 
First, consider $|z \cdot w| = | r_1e^{i\theta_1} \cdot r_2e^{i \theta_2}|$. Using basic algebraic laws, we can rewrite this equivalence as $|z \cdot w| = |r_1 r_2 e^{i(\theta_1 + \theta_2)} |$. Since any complex number $n = re^{i\theta}$ can be understood as $n = r\cos\theta + i r \sin\theta$, we may say that
\begin{equation*}
|z \cdot w| = |r_1 r_2 e^{i(\theta_1 + \theta_2)} | = |r_1 r_2\cos(\theta_1 + \theta_2) + i r_1 r_2 \sin(\theta_1 + \theta_2) |.
\end{equation*}
Then, using the definition of modulus, we reveal that that 
\begin{equation*}
|z \cdot w| = \sqrt{(r_1 r_2\cos(\theta_1 + \theta_2))^2 +  (r_1 r_2 \sin(\theta_1 + \theta_2))^2}.
\end{equation*}
Once again employing basic algebraic laws, we can rewrite the equivalence statement as 
\begin{equation*}
|z \cdot w| = r_1 r_2 \sqrt{\cos^2(\theta_1 + \theta_2) +  \sin^2(\theta_1 + \theta_2)}.
\end{equation*}
Next, we use the Pythagorean identity that $\cos^2\theta + \sin^2\theta = 1$ for all values of $\theta$. Thus, we have that
\begin{equation*}
|z \cdot w| = r_1 r_2 \sqrt 1 = r_1 r_2.
\end{equation*}
We will show that the same is true for $|z| \cdot |w|$. Substituting for polar notation, we write that $|z| \cdot |w| = |r_1e^{i\theta_1} | \cdot |r_2e^{i \theta_2}|$. Then, 
\begin{equation*}
|z| \cdot |w| = |r_1e^{i\theta_1} | \cdot |r_2e^{i \theta_2}| = | r_1\cos\theta_1 + i r_1\sin\theta_1| \cdot | r_2\cos\theta_2 + i r_2 \sin\theta_2|.
\end{equation*}
Employing the definition of modulus, we see that 
\begin{equation*}
|z| \cdot |w| = \sqrt{(r_1\cos\theta_1)^2 + (r_1\sin\theta_1)^2} \cdot \sqrt{(r_2\cos\theta_2)^2 + (r_2 \sin\theta_2)^2}.
\end{equation*}
Isolating $r_1$ and $r_2$ using algebraic laws, we see that 
\begin{equation*}
|z| \cdot |w| = r_1\sqrt{ \cos ^2 \theta_1 + \sin ^2 \theta_1} \cdot r_2\sqrt{ \cos ^2 \theta_2 +  \sin ^2 \theta_2}.
\end{equation*}
Finally, employing the Phythagorean identity, we see that 
\begin{equation*}
|z| \cdot |w| = r_1\sqrt{ 1} \cdot r_2\sqrt{1} = r_1 r_2.
\end{equation*}
Thus, because they are both equivalent to $r_1 r_2$, we have that $|z \cdot w| = |z| \cdot |w|$ for complex numbers $z$ and $w$, as desired.
\begin{flushright}
$\square$
\end{flushright}
\end{parts}

% QUESTION 4 ---------------------------------------------------
\question
For the parts below, let $z = a+bi$ and $w = c+di$ be complex numbers.
\begin{parts}
% QUESTION 4 PART A
\part
\textbf{Proposition.} $\overline{z+w} = \overline z + \overline w$.
\\
\\\textbf{Proof.} We will show that $\overline{z+w} = \overline z + \overline w$ by demonstrating that both $\overline{z+w}$ and $\overline z + \overline w$ are equal to $a+c - i(b+d)$. 
\par
First, consider 
\begin{equation*}
 \overline{z+w} = \overline{(a+bi)+(c+di)} = \overline{a+c + i(b+d)}.
\end{equation*}
By the definition of a conjugate, $\overline{z+w}$ is equivalent to the real part minus the imaginary part of $z+w$. In other words, $\overline{z+w} = a+c -i(b+d)$. Now, consider $\overline z + \overline w$. Substituting, we have  
\begin{equation*}
\overline z + \overline w = \overline {a+bi} + \overline {c+di}.
\end{equation*}
By the definition of a conjugate, we can write that $\overline z + \overline w = (a-bi)+(c-di)$. Rearranging to separate the real and imaginary parts, we have that 
\begin{equation*}
\overline z + \overline w = a+c - i(b+d). 
\end{equation*}
Thus, because they are both equivalent to $a+c - i(b+d)$, we have shown that $\overline{z+w} = \overline z + \overline w$, as desired.
\begin{flushright}
$\square$
\end{flushright}
% QUESTION 4 PART B
\part
\textbf{Proposition.} $\overline{z \cdot w} = \overline z \cdot \overline w$.
\\
\\\textbf{Proof.} We will show that $\overline{z \cdot w} = \overline z \cdot \overline w$ by demonstrating that both $\overline{z \cdot w}$ and $\overline z \cdot \overline w$ are equal to $ac - bd -i(bc+ad)$. First, consider $\overline{z \cdot w}$. Substituting for Cartesian form, we have 
\begin{equation*}
\overline{z \cdot w} = \overline{(a+bi) \cdot (c+di)}. 
\end{equation*}
Foiling, we find that 
\begin{align*}
\overline{z \cdot w} &= \overline{ac + ibc + iad -bd}; \\
 \overline{z \cdot w} &= \overline{ac  -bd+ i(bc + ad)}.
\end{align*}
Then by definition of the complex conjugate, we know that 
\begin{equation*}
\overline{z \cdot w} = ac  -bd - i(bc + ad). 
\end{equation*}
Now, consider $\overline z \cdot \overline w$. Thus, we have, 
\begin{equation*}
\overline z \cdot \overline w = \overline {a+bi} \cdot \overline {c+di}.
\end{equation*}
Using the the definition of a complex conjugate and foiling, we know that this is also equivalent to 
\begin{equation*}
\overline z \cdot \overline w = (a-bi) \cdot (c-di) =ac +ibc + iad -bd.
\end{equation*}
Then, with rearrangement of real and imaginary parts, we can conclude that 
\begin{equation*}
\overline z \cdot \overline w = ac - bd -i(bc+ad).
\end{equation*}
Since both $\overline{z \cdot w}$ and $\overline z \cdot \overline w$ are equivalent to $ac - bd -i(bc+ad)$, we have that $\overline{z \cdot w} = \overline z \cdot \overline w$, as desired.
\begin{flushright}
$\square$
\end{flushright}
% QUESTION 4 PART C
\part
\textbf{Proposition.} $\overline{z^n} = \left(\overline z\right)^n$ for any natural number $n \in \mathbb N$.
\\
\\\textbf{Proof.} Consider a natural number $n \in N$. We will show that $\overline{z^n} = \left(\overline z\right)^n$ where $n \geq 0$. Recall the theorem proved in (b) above, where  $\overline{z \cdot w} = \overline z \cdot \overline w$ for complex numbers $z,w \in \mathbb{C}$. If $n=0$, then it is true that $\overline{z^0} = \left(\overline z\right)^0$ because a complex number raised to the 0th power is 1, and $\overline 1 = 1$ because 1 is a real number (i.e., its imaginary part is 0). 
\par
Having shown that the statement $\overline{z^n} = \left(\overline z\right)^n$ is true for the base case $n=0$, we will now use an inductive assumption. Assume that $\overline{z^n} = \left(\overline z\right)^n$ is true for some $n$. We will use proof by induction to show that this is also true for $n+1$ so that  $$\overline{z^{n+1}} = \left(\overline z\right)^{n+1}.$$ 
Having assumed $\overline{z^n} = \left(\overline z\right)^n$, then let us consider the case for $n+1$, where $$\overline{z^{n+1}} = \overline{z^{n} \cdot z}.$$ 
Using the property found in (b) above, we can write that $\overline{z^{n} \cdot z} = \overline{z^{n}} \cdot \overline z$. We can then continue this process, separating $z$ to find the following pattern:
\begin{align*}
\overline{z^{n+1}} &= \overline{z^{n}} \cdot \overline z \\
&= \overline{z^{n-1}} \cdot \overline z  \cdot \overline z \\
&= \overline{z^{n-2}} \cdot \overline z  \cdot \overline z  \cdot \overline z\\
&= \overline{z^{n-3}} \cdot \overline z  \cdot \overline z  \cdot \overline z \cdot \overline z, \\
\end{align*}
until we have reached 
\begin{align*}
\overline{z^{n+1}} &= \overline{z^{0}} \cdot \left(\overline z\right)^{n+1} \\
&= \overline 1 \cdot \left(\overline z\right)^{n+1} \\
 &= 1 \cdot \left(\overline z\right)^{n+1} .
\end{align*}
Thus, we have shown that $\overline{z^{n+1}} = \left(\overline z\right)^{n+1}$. By inductive assumption, we may now conclude that  $\overline{z^n} = \left(\overline z\right)^n$ for any natural number $n \in \mathbb N$, as desired.
\begin{flushright}
$\square$
\end{flushright}
% QUESTION 4 PART D
\part
\textbf{Proposition.} Consider the following polynomial $p(z)$ with \emph{real coefficients}:  $$p(z) = \alpha_nz^n + \alpha_{n-1}z^{n-1} + \cdots + \alpha_1 z + \alpha_0,$$ where each $\alpha_i$ is a real number.  If a complex number $w$ is a root to the above polynomial with real coefficients, then its conjugate $\overline w$ is also a root to the same polynomial.  That is, if $p(w) = 0$, then $p(\overline w) = 0$.
\\
\\\textbf{Proof.} We will show that if $p(w) = 0$, then $p(\overline w) = 0$. Using sigma notation, we can write that 
\begin{equation*}
p(z) = \sum\limits_{k=0}^n a_k z^k .
\end{equation*}
Thus, having assumed a $ w \in \mathbb C$ such that $p(w) = 0$, we substitute for $w$ to write that 
\begin{equation*}
p(w) = \sum\limits_{k=0}^n a_k w^k = 0.
\end{equation*}
Note that with the above assumption, we can also write that 
\begin{equation*}
\overline{ \sum\limits_{k=0}^n a_k w^k} = \overline 0.
\end{equation*}
Now, consider $p(\overline w)$ such that 
\begin{equation*}
p(\overline w) = \sum\limits_{k=0}^n a_k \overline w^k.
\end{equation*}
Using the previously shown theorems of (a) - (c), we will show that $p(\overline w) = 0$. First, consider the concept shown in (c), in which $\overline{z^n} = \left(\overline z\right)^n$ for natural number $n$ and complex number $z$. Applying this, we see that 
\begin{equation*}
p(\overline w) = \sum\limits_{k=0}^n a_k \overline w^k = \sum\limits_{k=0}^n a_k \overline {w^k}. 
\end{equation*}
Now, recall that when $z$ is a real number, $\overline z = z$  because the imaginary part is equal to $0$. Therefore, since each $a_i$ is a real number, $a_i = \overline{a_i}$. Following, it is true that
\begin{equation*}
p(\overline w) = \sum\limits_{k=0}^n a_k \overline {w^k} = \sum\limits_{k=0}^n \overline{ a_k} \cdot \overline {w^k}, 
\end{equation*}
and we can apply the theorem in (b) where $\overline{z \cdot w} = \overline z \cdot \overline w$ for $z,w \in \mathbb C$. Thus we may write that
\begin{equation*}
p(\overline w) = \sum\limits_{k=0}^n \overline{ a_k} \cdot \overline {w^k} = \sum\limits_{k=0}^n \overline{ a_k w^k}. 
\end{equation*}
Finally, consider proof (a), which states that $\overline{z+w} = \overline z + \overline w$ for $z,w \in \mathbb C$. Because sigma notation is representation for repeated additions, we may write that 
\begin{equation*}
p(\overline w) =  \sum\limits_{k=0}^n \overline{ a_k w^k} =  \overline{\sum\limits_{k=0}^n  a_k w^k}. 
\end{equation*}
Since we already know that
\begin{equation*}
\overline{\sum\limits_{k=0}^n  a_k w^k} = \overline 0, 
\end{equation*}
and it is true that $0 = \overline 0$ because $0 \in \mathbb {R}$, then we have that 
\begin{equation*}
p(\overline w) = \overline{\sum\limits_{k=0}^n  a_k w^k} = 0.
\end{equation*}
In other words, $\overline w$ is a root of polynomial $p$. Therefore, if a complex number $w$ is a root to the polynomial $p$ with real coefficients, then its conjugate $\overline w$ is also a root to $p$, as desired.
\begin{flushright}
$\square$
\end{flushright}
\end{parts}

%Conclusion
\end{questions}
\end{document}
